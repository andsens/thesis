\chapter{Requirements}
The information lost after a view is rendered might not be useful and
the behavior of existing templating languages therefore inconsequential.
This is not the case. Let us consider a minimal template used for
displaying profile information:
\todo{\dots}
As you can see, the fields of the user object are printed into the HTML
at the appropriate places, leaving us with a normal page which can be
displayed in the browser.
Add the requirement that this form is not to be submitted via a
synchronous browser request but via AJAX. Now the developer has to
reverse engineer the generated HTML with JavaScript to obtain the
user information that is to be submitted. Any changes to the template
will now also require a change in client side code, particularly the
code, which finds the values in the form.
This example contains a rather obvious loss of information.
The position of the field attributes of the user object.
At the time of the rendering these positions are known, but as soon as the
result is reduced to simple a string that is sent to the client, this
information is lost.

\todo{Introduction to what I want to do about it}

\section{Development work flow}
The tool introduced in this thesis, will be developed via an iterative
work flow. Two or three versions of the solution to the requirements outlined
above will be discussed in this thesis. Each version building on the knowledge
acquired in the development process of the previous.
