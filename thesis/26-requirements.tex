\chapter{Requirements}
The data that fills the gaps in HTML templates often originates from the
database or another data source. This source is mapped on to an object graph
where in the case of a table the instance of a class represents a row in
said table.
The bindings of class fields to placeholders represent information in
themselves. This information about what object data in a HTML page originates
from is lost once the placeholders are filled with data and sent to the client.

This loss of information is inconsequential to the way web applications are
built with current frameworks, that does not however mean that it is useless.\\
Let us consider a minimal template used for displaying profile information:
\todo{\dots}
As you can see, the fields of the user object are printed into the HTML
at the appropriate places, leaving us with a normal page which can be
displayed in the browser.

The following scenario illustrates how this simple way of handling templates,
requires additional work when information about where data is put in the
template is not readily available:

\begin{shaded}
After the profile form has been styled with CSS, the developer decides that
the submission of the form should not issue a page reload. The tool of choice
for that is AJAX.

Once the form is working the way it should, everything is brought into
production. Metrics however suggest that changing the layout of the form would
increase usability.
The designer moves form fields around to make it easier for user to update
their profile.
All the while, the developer has to accommodate the design changes by modifying
the CSS selectors he uses to bind the form elements and the client-side version
of the user model together.
CSS IDs are used sparingly to avoid naming conflicts, so every correction
to a HTML template bears with it a correction in the selection of DOM nodes.
\end{shaded}
This example highlights a rather obvious loss of information, namely
the position of the form elements and their connection to model attributes.\\
When the placeholders of an HTML template are filled these positions are known,
but as soon as the result is reduced to simple a string that is sent
to the client, this information is lost.

The argument to uphold the status quo in this case is increased usage of
CSS IDs. The problem is however that this approach does not scale very well.
A complex naming scheme would be required to avoid naming collisions.
Every possibly modifiable DOM node would be tagged with an ID,
requiring more work in both the template writing and DOM binding of
client-side models.
\todo{Introduction to what I want to do about it}
