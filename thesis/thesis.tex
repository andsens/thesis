\documentclass[twoside,11pt,openright]{report}

\usepackage[latin1]{inputenc}
\usepackage[american]{babel}
\usepackage{a4}
\usepackage{latexsym}
\usepackage{amssymb}
\usepackage{amsmath}
\usepackage{epsfig}
\usepackage[T1]{fontenc}
\usepackage{lmodern}
\usepackage[labeled]{multibib}
\usepackage{color}
\usepackage{datetime}

\renewcommand*\ttdefault{txtt}

\newcommand{\todo}[1]{{\color[rgb]{.5,0,0}\textbf{$\blacktriangleright$#1$\blacktriangleleft$}}}

\newcites{A,B}{Primary Bibliography,Secondary Bibliography}

% see http://imf.au.dk/system/latex/bog/

\begin{document}

%%%%%%%%%%%%%%%%%%%%%%%%%%%%%%%%%%%%%%%%%%%%%%%%%%%%%%%%%%%%%%%%%%%%%%%

\pagestyle{empty} 
\pagenumbering{roman} 
\vspace*{\fill}\noindent{\rule{\linewidth}{1mm}\\[4ex]
{\Huge\sf Coupling Server-Side Templates and Client-Side Models}\\[2ex]
{\huge\sf Anders Ingemann, 20052979}\\[2ex]
\noindent\rule{\linewidth}{1mm}\\[4ex]
\noindent{\Large\sf Master's Thesis, Computer Science\\[1ex] 
\monthname\ \the\year  \\[1ex] Advisor: Michael Schwartzbach\\[15ex]}\\[\fill]}
\epsfig{file=logo.eps}\clearpage

%%%%%%%%%%%%%%%%%%%%%%%%%%%%%%%%%%%%%%%%%%%%%%%%%%%%%%%%%%%%%%%%%%%%%%%

\pagestyle{plain}
\chapter*{Abstract}
\addcontentsline{toc}{chapter}{Abstract}

\todo{in English\dots}

\chapter*{Resum\'e}
\addcontentsline{toc}{chapter}{Resum\'e}

\todo{in Danish\dots}

\chapter*{Acknowledgements}
\addcontentsline{toc}{chapter}{Acknowledgments}

\todo{\dots}

\vspace{2ex}
\begin{flushright}
  \emph{Anders Ingemann,}\\
  \emph{Aarhus, \today.}
\end{flushright}

\tableofcontents
\pagenumbering{arabic}
\setcounter{secnumdepth}{2}

%%%%%%%%%%%%%%%%%%%%%%%%%%%%%%%%%%%%%%%%%%%%%%%%%%%%%%%%%%%%%%%%%%%%%%%

\chapter{Introduction}
\label{ch:intro}

\todo{\dots}

\todo{example of a citation to primary literature: \citeA{lazypropagation2010},
and one to secondary literature: \citeB{ambiguity2010}}

%%%%%%%%%%%%%%%%%%%%%%%%%%%%%%%%%%%%%%%%%%%%%%%%%%%%%%%%%%%%%%%%%%%%%%%

\chapter{Developing webapplications}
Webapplications are on the rise. Not a day goes by where a new
webapp is popping up for uses that were previously reserved for a
program locally installed on a computer. Even more so: Previously
unimagined uses for any internet enabled device seem to be developed
at a rate that surpasses the former.

\section{What are webapplications?}
There are many definitions for a webapplication. What they all have in
common is the fact that they communicate with the outside world. Their
implementations differ from device to device, desktop PCs included.

In this thesis however, we will focus on webapplications which use a
modern web browser and with it HTML as their basis (HTML5 in particular).
Interactivity is supplied by JavaScript in this case.
There do exist other languages like Dart, CoffeeScript and Google Web Toolkit.
However, they are all translated into JavaScript if cross-browser
compatibility is a requirement (which it almost always is).

Another way to render a webapplication interactive, is by
rendering customized HTML pages on the server. This has both advantages
and disadvantages to the client-side scripting method.
data-processing speed
security concerns
client initialization
control over application (cross browser stuff)

\section{The development process}
The development process of a webapplication is similar to any other
software development process. One starts with the data to be modeled.
This might happen on the server and the client simultaneously.
A protocol for communication between the two is then established.
Even though the design layout for an application may have existed in
the beginning of the process it is usually only fully implemented when
most other critical components are in place.

\subsection{Usual design patterns}
The model-view-controller design pattern has proven itself to be a sane
choice for developing webapplications. Most frameworks today use this
pattern or variations thereof.
model-view-controller-presenter (/ModelView)
Designs of webapplications change with time, features are added or
removed and common processes simplified. In light of this, it is
desireable to ensure that the "View" part of the model-view-controller
pattern is easily modifieable.

%%%%%%%%%%%%%%%%%%%%%%%%%%%%%%%%%%%%%%%%%%%%%%%%%%%%%%%%%%%%%%%%%%%%%%%

\chapter{Templating languages}
\section{A survey}
\subsection{Mustache}
\subsection{\todo{\dots}}
\section{Limitations}
These templating languages are very different in their design.
All aim to improve one or more aspects of the templating task.
Of those Mustache seems to be specifically tailored for webapplications
with interactive JavaScript parts.
They do all have a common trait which in some cases can be an advantage
but given any specific implementation of a webapplication is a drawback:
They are completely oblivious of their surroundings. They draw the line
at the "View" part in order to encourage a seperation from the other
parts. This comes at the cost of lost information when sending a
rendered view to the client.

%%%%%%%%%%%%%%%%%%%%%%%%%%%%%%%%%%%%%%%%%%%%%%%%%%%%%%%%%%%%%%%%%%%%%%%

\chapter{Requirements}
The information lost after a view is rendered might not be useful and
the behavior of existing templating languages therefore inconsequential.
This is not the case. Let us consider minimal template used for
displaying profile information:
\todo{\dots}
As you can see, the fields of the user object are printed into the HTML
at the appropriate places, leaving us with a normal page which can be
displayed in the browser.
Add the requirement that this form is not to be submitted via a
synchronous browser request but via AJAX. Now the developer has to
reverse engineer the generated HTML with JavaScript to obtain the
user information that is to be submitted. Any changes to the template
will now also require a change in client side code, particularly the
code, which finds the values in the form.
This example contains a rather obvious loss of information.
The position of the field attributes of the user object.
At the time of the rendering these positions are known, but as soon as the
result is converted into a string that is sent to the client, this
information is lost.

\section{Development workflow}
\section{Initial prototype}
\subsection{Libraries}
\subsection{The application}
\subsection{Results}
\section{Plan for next iteration \todo{\dots}}

%%%%%%%%%%%%%%%%%%%%%%%%%%%%%%%%%%%%%%%%%%%%%%%%%%%%%%%%%%%%%%%%%%%%%%%

\chapter{Implementation}
\section{Templating language syntax}
\subsection{Integrating the parser}
\section{Template-aware clients}
\subsection{Client side architecture}
\subsection{Transmitting meta-data}

%%%%%%%%%%%%%%%%%%%%%%%%%%%%%%%%%%%%%%%%%%%%%%%%%%%%%%%%%%%%%%%%%%%%%%%

\chapter{Usage}
\section{Application example}

%%%%%%%%%%%%%%%%%%%%%%%%%%%%%%%%%%%%%%%%%%%%%%%%%%%%%%%%%%%%%%%%%%%%%%%

\chapter{Evaluation}
\section{Performance}
\section{Limitations}
\section{Advantages}
\subsection{Comparison}

%%%%%%%%%%%%%%%%%%%%%%%%%%%%%%%%%%%%%%%%%%%%%%%%%%%%%%%%%%%%%%%%%%%%%%%

\chapter{Future Work}

%%%%%%%%%%%%%%%%%%%%%%%%%%%%%%%%%%%%%%%%%%%%%%%%%%%%%%%%%%%%%%%%%%%%%%%

\chapter{Related Work}

%%%%%%%%%%%%%%%%%%%%%%%%%%%%%%%%%%%%%%%%%%%%%%%%%%%%%%%%%%%%%%%%%%%%%%%

\chapter{Conclusion}
\label{ch:conclusion}

\todo{\dots}

%%%%%%%%%%%%%%%%%%%%%%%%%%%%%%%%%%%%%%%%%%%%%%%%%%%%%%%%%%%%%%%%%%%%%%%

\addcontentsline{toc}{chapter}{Primary Bibliography}
\bibliographystyleA{plain} 
\bibliographyA{refs.bib}
\addcontentsline{toc}{chapter}{Secondary Bibliography}
\bibliographystyleB{plain} 
\bibliographyB{refs.bib} % remove this if you don't need secondary literature

\end{document}

