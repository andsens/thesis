\chapter{Tools of the trade}
In order to achieve a separation of responsibility various frameworks and tools
are at a developers disposal. In this thesis we will focus on two of them
specifically.
\section{Client frameworks}
\subsection{Backbone.js}
\subsection{XPath}
XPath is a language that allows us to define a path from a root node to another
node. Although limited, the language is fairly concise and directly built into
JavaScript. We will only be dealing with XPath 1.0, because version 2.0 is, with
the exception of Microsoft Internet Explorer, not implemented in any browsers
yet and likely will not ever be (the final specification was released 2006).
\section{Templating languages}
A templating language allows the developer to create HTML documents containing
placeholders, which later can be filled by a Model and its attributes.
Templating is part of the "View" component in the Model-View-Controller
pattern. In the following we will have a look at one such templating language.
\subsection{Mustache}
Mustache is a so called "logic-less" template engine.
This subtitle derives from the fact that there are no control flow statements
(e.g. if and else statements and while loops). Instead there are only tags.
Tags in in this context can be understood as an advanced form of placeholders.
Some tags are replaced with a string, some are replaced with nothing,
yet others are replace with series of strings or even more tags.
\subsubsection{Limitations}
\todo{fix entire paragraph}
These templating languages are very different in their design.
All aim to improve one or more aspects of the templating task.
Of those Mustache seems to be specifically tailored for web applications
with interactive JavaScript parts.
They all have a common trait which in some cases can be an advantage
but given any specific implementation of a web application is a drawback:
They are completely oblivious of their surroundings. They draw the line
at the "View" part in order to encourage a separation from the other
parts. This comes at the cost of lost information when sending a
rendered view to the client.
