% !TEX root = thesis.tex
\documentclass[thesis.tex]{subfiles} 
\begin{document}

\todo{http://rivetsjs.com/}

\todo{Discuss we accomplished what at first seemed to require a framework
but managed to convert it into a tool that has a much greater applicability}

\chapter{Related Work}
\section{MAWL}
Mawl\cite{MAWL} is a framework and domain-specific language developed in 1999
to aid the handling of HTML forms as well as telephone forms.
It does this by supplying the developer with a framework which compiles
a set of MAWL templates and ``sessions'' to executables
the browser can communicate with via the CGI on the web server.

The primary goal behind MAWL is to empower the developer with a better
organization of the data flow in form-based services.
This goal is achieved by creating a form abstraction language in which the
semantics of forms can be specified while the presentation layer of these forms
is handled by MAWL templates (MHTML).
CGI programs handling the data flow need no longer be programmed in
perl, Tcl or the Korn Shell, but can be compiled from said language.

A secondary advantageous property of MAWL is its ability to verify the
type signature of the form abstraction against structure of an MHTML template.
Although the paper does not go into detail explaining how these templates are
analyzed a parallel can be drawn to our Comb pre-parser tool which employs
Parsec to construct an abstract syntax tree. Much like MHTML is checked for
consistency, our parser also verifies mustache templates and displays any
possible errors\footnote{See \ref{sec:emergent}}.
\todo{ref to emergent props}
\begin{citequote}{\cite{MAWL}}
First, the sessions and the MHTML can be independently analyzed to ensure that
they are internally consistent.
For the sessions, this means standard type checking and semantics checking.
For MHTML templates, this means verifying that a template is legal MHTML.
\end{citequote}


\section{Template::Extract}
Template::Extract\cite{TPLEXTRACT} is a perl module with a functionality much
like Comb. It is written for the ``Template Toolkit'' templating language and
allows the developer to extract values from a rendered template when given the
template. This extraction is accomplished by compiling a regular expression
based on the template and applying it to the rendered template.

\begin{citequote}{\cite{TPLEXTRACT}}
Extraction is done by transforming the result from Template::Parser to a highly
esoteric regular expression, which utilizes the (?\{...\}) construct to insert
matched parameters into the hash reference.
\end{citequote}

Our revised goal exactly matches Template::Extract, Comb however operates on the
Document Object Model while Template::Extract operates on a single string.
In \ref{chap:arch} we evaluated that possibility but discovered that the strings
returned by the \inline{innerHTML} property of DOM elements differs between
browsers and does not reflect the original rendered template.
By extracting values from the DOM we also retrieve references to the nodes they
were extracted from\footnote{
	as detailed in \ref{sec:parent-nodes}
}. This additional piece of information allows Comb to become more than a simple
value extraction library.

\section{}
\cite{DYNDOC}

\cite{HASKELL}
\cite{WASH}
\cite{ML}

\end{document}
