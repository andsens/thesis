\chapter{First iteration - The initial prototype}
\label{chap:first}
First we make a rudimentary prototype. It is an exploratory prototype,
meaning none of its code is intended to be carried over into the next iteration.
The initial prototype is an interactive application for maintaining a movie
library. Movie details can be edited and actors can be added to that library.
It is not very useful in practice, but serves to make the basic idea more
concrete in the following ways:

\begin{itemize}
	\label{list:prototype-motivation}
	\item \emph{Materialize peripheral concepts}\\
	The prototype is meant to capture the core concept of the initial idea
	(to couple client models with server templates).
	Many of the less pronounced concepts of that idea will need to be made
	concrete in order for the core concept to work.
	\item \emph{Highlight logical errors}\\
	Edge cases of an idea can be crucial to its successful implementation.
	Problems involving those cases may have been erroneously dismissed as trivial.
	Some of those problems may not have solutions or workarounds, which
	means the work on the entire project has been in vain. By making a working
	prototype, those errors will be discovered early on.
	\item \emph{Discover additional requirements}\\
	The implementation of a movie library allows for practical challenges to
	arise, which might not have been discovered if we implemented such an
	application at the conclusion of the project. The advantage in the
	first iteration is our ability to course correct in subsequent iterations,
	if such errors should occur.
\end{itemize}

\section{Architecture}

The server-side back end is based on PHP and MySQL. The client-side uses
HTML5, JavaScript (+XPath) and CSS as its core technologies.

\subsection{Libraries}
In order to speed up the prototyping process a plethora of libraries have been
used. Excluding basic core technologies like JavaScript, MySQL and PHP, the
application stack consists of the following:
\begin{itemize}
	\item \emph{less}\\
	A superset of CSS providing variables, calculations and nested selectors. It
	is a JavaScript library which compiles included less files into CSS.
	\item \emph{jQuery}\\
	The de facto standard when creating web applications. Among other things it
	simplifies the interaction with the DOM.
	\item \emph{backbone.js}\\
	Backbone.js is a JavaScript Model/View framework. It provides the developer
	with View, Model and Collection prototypes. The View prototype can be
	considered analogous to the aforementioned ViewModel, while the Model and
	Collection part make up the Model component and collections thereof,
	respectively.
	\item \emph{php-activerecord}\\
	PHP ActiveRecord is the server-side library utilized to communicate with the
	database.
\end{itemize}

\subsection{Templating}
The application features rudimentary HTML templates, which are not backed by
any engine. Instead PHP is embedded directly into the HTML files.
Although PHP allows for more complex templates, we keep them simple in order to
place the same constraints on the templates as we would have when using
Mustache.
We convert data from the database into HTML by fetching it from the database and
by forwarding that data to the embedded PHP.
As an example, figure \ref{fig:movie.view.tpl} illustrates what the template
for a movie looks like.
None of the variables are scoped. Every variable can be referred to once it has
been initialized.
The PHP is embedded between \inline{<?php} and \inline{?>} tags.

\begin{figure}
	\label{fig:movie.view.tpl}
	\centering
	\lstinputlisting[language=HTML]{../proof_of_concept/view/types/movie.view.tpl}
	\caption{
		The file \inline{movie.view.tpl}.
		A template in the initial prototype.}
\end{figure}

\begin{itemize}
	\item \emph{Template inclusion}\\
	Starting from the root sub-templates are included via a simple
	PHP \inline{require} command.
	\item \emph{Simple variables}\\
	Simple variables are inserted via a PHP \inline{echo} command.
	\item \emph{Objects}\\
	Objects are converted into arrays so their fields can be initialized as
	variables with the \inline{extract} method.
	\item \emph{Collections}\\
	Collections (i.e. PHP arrays) are simply iterated through, in this prototype
	only objects are present in these arrays, the block inside the
	\inline{foreach} loop therefore simply contains the aforementioned method
	for placing object fields into the global scope.
	\item \emph{Client-side templates}\\
	There is a small portion of client-side templates, which are used whenever new
	data is added.  They do not have any impact on the concept explored in
	this prototype, instead they are an attempt to explore edge cases as outlined
	in the motivations for an exploratory prototype in the introduction of this
	chapter (\ref{list:prototype-motivation}).
\end{itemize}

\subsection{Models and ViewModels}
Every Model on the server-side is linked to the database.
One instance of a model represents one entity in the database.
Each of these Models is also represented on the client-side using backbone.js,
which provides us with a "Backbone.Model" base class that can be extended.
\todo{Note for clarity that with "class" in JavaScript
we actually mean object prototype}

Using the "Backbone.View", we can create ViewModels that bind the Model and the
DOM together. For example, this can be used to listen to changes in
form elements, which the ViewModel translates into changes of the corresponding
fields in the Model. The Model can in turn synchronize those changes to the
server.

Although not implemented in this prototype, the Model can also receive changes
from the server (via server push or client pull methods) and notify
the ViewModel about those changes. The ViewModel can then update the
DOM (the user interface) with those changes.

\subsection{Collections}
In addition the above mentioned base classes backbone.js provides a third class.
It is called a Collection and contains any number of other Model instances.
In a given Collection the models all have the same type.

\section{Coupling client-side and server-side models}
Models and ViewModels are powerful abstractions.
We can extend them to make use of the information that specifies
which server-side field attribute belongs with what content on the HTML page.

Since the client-side model mirrors the server-side model a direct mapping of
the information retrieved from the templates should be possible.
The classification of this information is of importance: We will need to know
whether a field contains a collection, a string, or an aggregated model,
in order to parse the DOM properly.

We have two possible abstractions this information can be attached to and
used by: The Model and the ViewModel.
The information specifies where a Model field is located in the DOM, which would
make the Model the optimal candidate. However, the information has a
localized context since there can exist more than one server-side template per
server-side model.
This is at odds with the fact that there is only one client-side Model
per server-side model.
On the other hand there can be more than one ViewModel per Model.
The ViewModels may even be coupled one to one with the templates.
This property renders the ViewModel better suited to tackle this problem.

Storing the information on the ViewModel and letting it utilize it is
advantageous because that information is valuable when binding event listeners
to the DOM or manipulating DOM nodes otherwise.

In this prototype we will not focus on retrieving the information from
the templates. Instead we assume this extraction has already taken place and
simply hard-code XPaths into the ViewModels. Each XPath points to a
position in the DOM where a server-side model field has been inserted.
The XPath is labeled with the name of that field.

The ViewModel is a means to an end: It does not enable any meaningful
interaction with the web application by simply binding to DOM nodes.
It does however function as a bridge between that DOM and the Model.
The Model in turn can communicate with the server, which can process
the user interaction and return a meaningful response.

\subsection{Parsing the DOM}

The Models that are to hold the data we want to handle need to be created and
populated with the data from the DOM when the page has loaded.
To that end we use the ViewModels to parse the HTML and create both them and
the Models they are attached to.
A recursive approach will simplify the parsing in this matter, since the DOM
is a tree structure.

We bootstrap the parsing function by giving it a "Root" model and a "RootView"
ViewModel. Both are prototype objects, that will later be instantiated.
The RootView has an XPath attached to it, which points to the list of movies
in the DOM. The parsing function returns an instantiated ViewModel with a Model
attached to it (e.g. the bootstrapping yields a "RootView" object containing
a "Root" object).

During the process every XPath in the ViewModel is examined.
Since any XPath is labeled with the name of the corresponding field on the
Model, we can query the Model for the type of that field
(this is hard-coded for the time being). The function \inline{getAttrType()}
returns that type.

We can follow three courses of action depending on the type
returned by the Model.
\begin{itemize}
	\item \emph{The field is a simple type}\\
	For strings, integers and the like we populate the field of the Model
	with that value, and proceed to the next XPath.
	\item \emph{The field aggregates another Model}\\
	The function queries the Model for the type of Model its field aggregates
	(\inline{getComplexType()}).
	We the recurse by calling the parsing function again.
	This time it is called with the aggregated Model class and
	the ViewModel class, which is returned by the \inline{view()} function
	we attached to the XPath.
	\todo{This highlights that we need some way of mapping templates to viewmodels}
	\item \emph{The field is a Collection}\\
	We query the Model for the type of Model the collection contains.
	The XPath can return more than one DOM node.
	For each of these nodes, we recurse.
	The return value of the function is pushed on to the Collection.
\end{itemize}

Once all XPaths have been examined, the function instantiates the Model class
that was passed to it in the beginning. It is populated with the field values
collected while examining the XPaths.
The ViewModel, also a function argument, is then instantiated with
the Model instance as one of the arguments.
This ViewModel is the return value of the function.

One drawback to the method we use to obtain field values is the requirement
for a post-processing function, that takes an XPath result and
returns the correct value of a field.
This is necessary because XPath is not an exact query language.
\begin{itemize}
	\item DOM node attributes will be returned with both their attribute name and
	their value as one string. This is undesirable, since we almost always only
	place a server-side field value into the value part of a DOM node attribute.
	\item Substrings can not be retrieved with XPath, it can only return entire
	text nodes.
\end{itemize}

\section{Results - Plans for next iteration}
\label{sec:first:results}
In the introduction of this chapter we listed some motivations for making
this prototype.

\begin{itemize}
	\item \emph{Materialize peripheral concepts}\\
	We have created a movie database that supports a simple interface
	for maintenance and browsing interaction.
	Through this process we have discovered the recursive nature of retrieving
	model field values from the DOM.
	Less pronounced concepts like the classification of model field types have 
	been made more concrete.
	\item \emph{Highlight logical errors}\\
	We have not uncovered any major logical errors that would require us to
	rethink the idea for coupling client-side models to server-side templates.
	\item \emph{Discover additional requirements}\\
	We not only mapped fields of models that aggregated another single model,
	but also collections of models. We solved this challenge in the prototype,
	by simply iterating through the nodes and adding them to
	a backbone collection. This method will need to be refined in the following
	iteration.
	We also discovered another major requirement, which we elaborate upon in more
	detail in the following paragraph.
\end{itemize}

The function \inline{getComplexType()} illustrates, that we will need some form
of mapping between ViewModels and templates.
It will be tedious and error prone for the developer to create those mappings
by hand. A process, which automates the coupling of the ViewModels with
server-side templates will be needed in the next iteration.

While developing this application various libraries have been examined for their
viability. backbone.js has in this case proven itself to be a very good fit.
Its View prototype is made to bind with the interface while being Model aware.
Such a component is what is needed to put the information about
the placement of data in the DOM to good use.
