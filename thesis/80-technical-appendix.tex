% !TEX root = thesis.tex
\documentclass[thesis.tex]{subfiles} 
\begin{document}

\chapter{Technical Appendix}

\section{Downloads}
\label{app:downloads}
The prototype application from chapter \ref{chap:prototype} and the Comb
pre-parser and client library can be downloaded together with
the demo applications from chapter \ref{chap:demo} as a single gzipped tarball
from the URL: \url{X}

% Since this PDF originated from the archive, a SHA-1 sum cannot be included.
% Refer to the printed edition to verify the contents of the archive.
The SHA-1 sum of the archive is: \inline{X}

\todo{Add information about where to download the prototype, Comb and demo applications
also include all three editions of the thesis}

\section{Digital version of this thesis}
This thesis exists as PDFs in three different versions:
Print (adjusted grayscale for better contrast),
digital (colors and single sided layout) and
ebook (adjusted grayscale with dimensions adjusted for the Amazon Kindle).

All versions are included in the archive from appendix \ref{app:downloads}

\section{Bootstrap}
\label{app:bootstrap}
Bootstrap is a combination of JavaScript and CSS which supplies a developer
with pre-styled components to build a web application. Colors and shapes
have been tuned to increase user-friendliness.
The website \url{http://twitter.github.com/bootstrap/} itself uses the package.
The blue gradient buttons are easily recognizable and can be found in many
websites across the web.

\section{chaplin}
\label{app:chaplin}

Chaplin is a new client-side framework, which was created in February 2012.
The motivation behind it was to create a framework that allows developers to
follow a set of conventions more easily. Backbone.js has both views and models
(and routes, for controllers), but does not force any specific way of
structuring code. In this respect Backbone.js can be seen more as a tool than a
framework.
Chaplin extends the models and views from Backbone.js and adds more features.
It introduces concepts such as ``subviews'' - views that aggregate other views.
This allows the developer among other things to better mirror the structure of
the DOM.

The framework also allows the developer to use any template engine he desires.
The engine simply needs to return an object, which jQuery can append to the
wrapping DOM element of the view.

A very useful feature of Chaplin is the automatic memory management.
When creating single page web applications, the developer has to dispose each
view manually. This challenge is best illustrated with regard to event handlers.
Event handlers are functions, that are called when an event on a DOM node or
an other object is triggered. Often this function manipulates and accesses
properties stored on a view. To allow for this access, the function stores a
pointer to the view via a closure. Since the function is stored with the
DOM node or object on which it is listening for events, any view the developer
wants to dispose needs to stop listening on those events as well.
Chaplin unbinds these event handlers for the developer when the view is
disposed, allowing the browser to free up memory.

\section{TagSoup}
\label{app:tagsoup}
TagSoup is a library for parsing HTML and XML. Its documentation is available at
\url{http://hackage.haskell.org/package/tagsoup-0.12.8}

\section{Node.js}
\label{app:nodejs}
NodeJS is built on the JavaScript runtime for the Chrome browser and allows
execution of JavaScript outside the browser. It complements JavaScript with
additional APIs to access resources like the file system.
It is available at \url{http://nodejs.org/}

\end{document}
