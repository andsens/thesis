% !TEX root = thesis.tex
\documentclass[thesis.tex]{subfiles} 
\begin{document}

\chapter{Introduction}
\label{chap:intro}
Web applications use templates to abstract data from the presentation layer.
Most template engines work by replacing placeholders with corresponding values
in a dataset.
Once a template has been rendered the information about the location of these
placeholders is discarded and the HTML is sent to the client.
On the client, the information required to bootstrap an application structure
with data may be embedded in the HTML. Currently the developer accesses such
information by leveraging the structure of the template to create selectors
pinpointing its location. Actionable parts of the HTML are designated the same
way.

We propose a tool that utilizes this otherwise discarded information to parse
the HTML on the client-side and extract the original dataset, while preserving
the locations whence any value was extracted. The dataset allows the developer
to interact more naturally with the HTML and makes the client-side application
less fragile in regard to template changes.

By basing our tool on an existing templating language and not introducing
extraneous requirements we distinguish it from existing solutions,
which require the use of their own templating language or only solve the problem
in a specific web server programming language.


Background: We can actually use the discarded information for cool things.

frameworks go the easy way and solve it by conquering both sides

templatestuff fragile to changes

Albeit the field of web application development is young and the first web
application (Viaweb\footnote{ostensibly this was the
first \emph{prominent} web application \todo{citation needed?}}) was conceived of in 1995,
it has over the years received a potent influx of novel ideas and solutions
to deal with the challenges of client-server architectures.

\todo{Talk about frameworks}

In this thesis we will develop a tool to increase the cohesion of the client and
the server by enabling the client to extract values from server-side rendered
HTML-templates. We will accomplish this feat this by analyzing templates and
parse rendered templates based on this analysis.

In chapter \ref{chap:webapps} we will examine either side of the client-server
paradigm and see how both can work in concert to exceed the sum of their parts.

Chapter \ref{chap:requirements} explains why extracting values from rendered
templates is a worthwhile endeavor and couples this task to the process of
binding these values directly to a data representation available on the
client-side.

We will develop an exploratory prototype in chapter \ref{chap:prototype} to
discover additional requirements, highlight any conceptual errors about our
goal and explore the entire set of challenges this goal encompasses.

In chapter \ref{chap:mustache} we introduce Mustache;
The templating language we will base our solution on. We will explain in detail
its syntax and capabilities.


Chapter \ref{chap:revised}




\todo{write}

\todo{Doublecheck if "we" is used correctly throughout the thesis}
\todo{developer vs. user, only one of the terms should be used}
\todo{We moved the mustache chapter, so we will need to introduce mustache a little bit here}

\end{document}
