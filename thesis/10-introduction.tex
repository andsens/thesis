% !TEX root = thesis.tex
\documentclass[thesis.tex]{subfiles} 
\begin{document}

\chapter{Introduction}
\label{chap:intro}
The field of web application development is young; the first prominent
web application (Viaweb) was conceived as late as in 1995.
Over the years the field has received a potent influx of novel ideas and
solutions to deal with the challenges of client-server architecture.
Grounded in the dynamic nature of web applications, developers use templates to
abstract data from the presentation layer.
Most template engines work by replacing placeholders in templates with
corresponding values from a dataset.
Once a template has been rendered the information about the location of these
placeholders is discarded and the HTML is sent to the client.
On the client, the information required to bootstrap an application structure
with data may be embedded in the HTML. Currently the developer accesses such
information by leveraging the structure of the template to create selectors
pinpointing its location. Actionable parts of the HTML are designated in the
same way.

We propose a tool that utilizes this otherwise discarded information to parse
the HTML on the client-side and extract the original dataset, while preserving
the locations whence any value was extracted. The dataset allows the developer
to interact more naturally with the HTML and makes the client-side application
less fragile in regard to template changes.

By basing our tool on an existing template language, externalizing their
analysis and not introducing extraneous requirements we distinguish it from
existing solutions such as \todo{},
which require the use of their own template language or only solve the problem
for one specific web server programming language.


This thesis is divided into three parts. First we delve into the structure
and development of web applications (chapter \ref{chap:webapps}).
Based on what we have uncovered we then go on to specify
requirements (chapter \ref{chap:requirements}) for our
exploratory prototype (chapter \ref{chap:prototype}).
Once we have analyzed the results and revised our
requirements (chapter \ref{chap:revised}) we proceed to the second part of the
thesis in which we devise an architecture (chapter \ref{chap:architecture}) for
the subsequent implementation (chapter \ref{chap:implementation}) of our tool.
We also build two demonstration applications (chapter \ref{chap:demo}) that each
employ our tool in different ways.
In the third part we evaluate our tool (chapter \ref{chap:evaluation}) and
compare it to related work (chapter \ref{chap:related}), on these notes we then
conclude the thesis (chapter \ref{chap:conclusion}).

\end{document}
