% !TEX root = thesis.tex
\documentclass[thesis.tex]{subfiles} 
\begin{document}

\chapter{Introduction}
\label{chap:intro}
Web applications use templates to abstract data from the presentation layer.
Most template engines work by replacing placeholders with corresponding values
in a dataset.
Once a template has been rendered the information about the location of these
placeholders is discarded and the HTML is sent to the client.
On the client, the information required to bootstrap an application structure
with data may be embedded in the HTML. Currently the developer accesses such
information by leveraging the structure of the template to create selectors
pinpointing its location. Actionable parts of the HTML are designated the same
way.

We propose a tool that utilizes this otherwise discarded information to parse
the HTML on the client-side and extract the original dataset, while preserving
the locations whence any value was extracted. The dataset allows the developer
to interact more naturally with the HTML and makes the client-side application
less fragile in regard to template changes.

By basing our tool on an existing template language and not introducing
extraneous requirements we distinguish it from existing solutions,
which require the use of their own template language or only solve the problem
in a specific web server programming language.\todo{More about frameworks}


This thesis is divided into three parts. First we delve into the structure
and development of web applications (\ref{chap:webapps}). Based on what we have
uncovered we then go on to create requirements (\ref{chap:requirements}) for our
exploratory prototype (\ref{chap:prototype}).
Once we have analyzed the results and revised our requirements
(\ref{chap:revised}) we proceed to the second part of the thesis in which we
devise an architecture (\ref{chap:architecture}) for the subsequent
implementation (\ref{chap:implementation}) of our tool. We also build two
demonstration applications (\ref{chap:demo}) that employ our creation in very
different ways.
In the third part we evaluate our tool (\ref{chap:evaluation}) and
compare it to related works (\ref{chap:related}), on these notes we will then
conclude the thesis (\ref{chap:conclusion}).

\end{document}
