% !TEX root = ../thesis.tex
\newcommand{\drawheight}{6}
\newcommand{\boxheight}{\drawheight/3}
\newcommand{\lnlength}{\textwidth/7}
\newcommand{\commlength}{\textwidth/7/4}
\begin{tikzpicture}
	\node[fill=lgray,
		corners={(0,0) (\textwidth,\boxheight)}]
		(serverbg) {};
	\node[below=0.0 of serverbg.north] {\sffamily\huge\color{gray}{server}};
	
	\node[fill=lgray,draw=lgray,
		corners={(0,\boxheight*2) (\textwidth,\drawheight)}]
		(clientbg) {};
	\node[above=0.0 of clientbg.south] {\sffamily\huge\color{gray}{client}};
	
	\node[corners={(0,\boxheight) (\textwidth,\boxheight*2)}]
		(timebg) {};
	\node[below=0.0 of clientbg.south west,anchor=north west] {\sffamily\huge\color{gray}{time}};
	
	
	\draw[->,fill=lgray,draw=lgray,line width=5pt]
		let \p1 = (timebg.west) in (0,\y1) -- (\textwidth,\y1);
	
	
	\draw[->,line width=2pt]
		let \p1 = (clientbg.west) in
			(\x1,\y1) node (C1) {}
		--
			(\lnlength,\y1) node[pos=0.5,above]{user activity} node (C1') {};
	
	\draw[->,line width=2pt]
		let \p1 = (clientbg) in
			(\x1-\lnlength/2,\y1) node (C2) {}
		--
			(\x1+\lnlength/2,\y1) node[pos=0.5,above]{user activity} node (C2') {};
	
	\draw[->,line width=2pt]
		let \p1 = (clientbg.east) in
			(\x1-\lnlength,\y1) node (C3) {}
		--
			(\x1,\y1) node[pos=0.5,above]{user activity} node (C3') {};
	
	
	
	\draw[->,line width=2pt]
		let \p1 = (serverbg), \p2 = (C1'), \p3 = (C2) in
			(\x2+\commlength,\y1) node (S1) {}
				--
			(\x3-\commlength,\y1) node[pos=0.5,below]{server processing} node (S1') {};
	
	
	\draw[->,line width=2pt]
		let \p1 = (serverbg), \p2 = (C2'), \p3 = (C3) in
			(\x2+\commlength,\y1) node (S2) {}
		--
			(\x3-\commlength,\y1) node[pos=0.5,below]{server processing} node (S2') {};
	
		
	\draw[->] (C1') -- (S1) node[pos=0.5,above,sloped] {query};
	\draw[->] (S1') -- (C2) node[pos=0.5,above,sloped] {response};
	\draw[->] (C2') -- (S2) node[pos=0.5,above,sloped] {query};
	\draw[->] (S2') -- (C3) node[pos=0.5,above,sloped] {response};
	
\end{tikzpicture}
