% !TEX root = thesis.tex
\documentclass[thesis.tex]{subfiles} 
\begin{document}

\chapter{Evaluation}
\label{chap:evaluation}

\todo{stable to changes by making it a tool: rule of robustness}

In this chapter we will assess the usefulness of Comb in light of the
demo applications from chapter \ref{chap:demo}.
Following this, we will analyze whether we achieved the goals we set ourselves.
We will also examine potential improvements of Comb and discuss its shortcomings
as well.




\section{Assessment of Comb}
\label{sec:assessment}

\subsection{Movie Database \#2}
\label{sec:asses-mdb}
The MovieDatabase \#2 has highlighted various advantages of using Comb over
normal jQuery selectors to extract values from rendered templates.
The transition between two very different templates is seamless when using Comb,
where none of the identifiers we used to retrieve values from the dataset
required any modification. This stands in stark contrast with the jQuery version
of our demo, where in fact none of the selectors stayed the same.

In our small example this need for adjustment may seem inconsequential.
However, when projected onto larger projects it is easy to imagine how this
process may become a daunting task: First the developer will need to locate all
Views the change affects. Following that, he will in his mind have to construct
the DOM tree of the template in question to check whether a given selector
requires adjustment.
If an adjustment is required, he will also need to make sure that
it captures only the elements that he intends to capture.
The template may even need to be changed because matching only a specific
element is impossible or too complex.
During all of this the developer has to switch back and forth between
the template and the view. With Comb, this process is not necessary, the
developer (or template designer) simply runs our pre-parser tool to refresh the
comb file and continues with his next task.

Consider also how the id of a movie was fetched in the jQuery version
(figure \ref{fig:jquery-fetch-substring}). The id is embedded in an element
attribute with a prefix. To arrive only at the id, we had to strip away the
``movie-'' part of the attribute by discarding the first 6 characters of the
``id'' attribute.
In an era where first-person shooters can be played directly in the browser
and processors are manufactured on the nanometer scale, counting characters
should not be part of the daily routine of a web-developer.
\begin{figure}
	\centering
	\begin{lstlisting}
@model.set 'id', (@$('>details').attr 'id').substring 6
	\end{lstlisting}
	\caption{Retrieving the id of a movie with jQuery}
	\label{fig:jquery-fetch-substring}
\end{figure}
Looking at the Comb alternative in figure \ref{fig:comb-fetch-substring}, we can
see that Comb already extracted the id from the attribute and discarded the
``movie-'' prefix.
We did not even have to look up where the value was placed in the template.
\begin{figure}
	\centering
	\begin{lstlisting}
@model.set 'id', @data.id.value
	\end{lstlisting}
	\caption{Retrieving the id of a movie with Comb}
	\label{fig:comb-fetch-substring}
\end{figure}
\todo{Talk about how we offload getAttrType from the prototype to the ModelView implementation}
This kind of dataset extraction is also visible when we examine the list of
movies in the bootstrap version of our application in figure
\ref{fig:movielist.mustache}.
\begin{figure}
	\centering
	\begin{lstlisting}[language=mustache]
<ul class="nav nav-list">
	<li class="nav-header">Movies</li>
	{{#movies}}<li><a data-toggle="tab" href="#movietab-{{id}}">{{title}}</a></li>{{/movies}}
	<li class="nav-header"><a href="#" id="add-movie"><i class="icon-plus"></i></a></li>
</ul>
	\end{lstlisting}
	\caption{Tab buttons for the list of movies in the bootstrap version of the Movie Database}
	\label{fig:movielist.mustache}
\end{figure}
In our application we did not need to extract data
from this specific part of the template, since the subsequent HTML for the
actual tab-panes contained the same and additional data.
If we were to determine the number of movies using this list with jQuery,
we would count the number of list items it contains. Subsequently we will need
to subtract 2 from this number to account for the additional list headers.
Alternatively we could add a class name to the list item in the
\inline{\{\{\#movies\}\}} loop, making it distinguishable from the header list
items.
Figure \ref{fig:movielist-jquery.js} illustrates these possible selectors.
\begin{figure}
	\centering
	\begin{lstlisting}
@$('ul>li').length-2

// requires adding a "movie" class to the list items
@$('ul>li.movie').length

// Additional possibility
@$('ul>li[class!="nav-header"]').length
	\end{lstlisting}
	\caption{Three selectors to count the number of movies in \ref{fig:movielist.mustache}}
	\label{fig:movielist-jquery.js}
\end{figure}

With Comb it is possible to retrieve such \emph{implicit} values quite easily:
We simply access the \inline{movies} array in the extracted dataset.
In addition to this strategy being simpler we also conserve the meaning behind
our code. Any developer looking at the code will immediately understand what
the number from the \inline{length} property represents without having to
examine the associated template. This holds true especially for the first
selector of figure \ref{fig:movielist-jquery.js} where the magic number "2"
seems to appear out of nowhere.
\begin{figure}
	\centering
	\begin{lstlisting}
@data.movies.length
	\end{lstlisting}
	\caption{Counting the number of movies with Comb}
	\label{fig:movielist-comb.js}
\end{figure}
Adding classes to the template for the sole purpose of being able to extract
data from the rendered template like we do in the second selector of figure
\ref{fig:movielist-jquery.js} is not a sound strategy. We pollute the CSS
selector namespace with classes that have nothing to do with the layout and
design of the web application.

Some scenarios may however require us to do exactly that, because there is no
section surrounding a group of tags. We can however extend Comb to also
recognize mustache comments, they do not change a rendered template in any way,
but give the developer a possibility to annotate parts of a template. In section
\ref{sec:comment-selectors} we examine this extension in more detail.

Annotations like these could also be used to take advantage of the
\inline{parentNode} property we return for all variables. In the Comb version
of our Movie Database we still used jQuery selectors to access parts of the HTML
where we wanted to listen for click events from the user. Specifically we wanted
to take action, when the plus-signs for adding a movie or a new role in a movie
were clicked.
Instead of relying on those selectors we could have added a variable to an
attribute of the button and accessed the \inline{parentNode} property.
The identifier of that variable should of course not reference any identifier we
might expect in the dataset, since we only intend to use it to access its parent
node and not its value.
In light of this use-case a more sound approach would indeed be to use mustache
comments instead.

The \inline{parentNode} property is used in our Movie Database for another
purpose: We update the properties of a movie Model while the user is editing it.
To bind event listeners to the nodes containing the editable text we wrap the
node referenced from the dataset in a jQuery object and leverage its event
abstraction layer to update the corresponding property on the model on ``keyup''
events. We abstracted this process and added it to the abstract View object
all our Views extend, the task to make an element editable then becomes very
simple: \inline{@editable('title')} is all it takes. Since the identifier in the
dataset is identical to the property name on the Model, \inline{editable()}
requires only one parameter to listen to changes on the parent node, extract the
new value from it and update the corresponding model property.

The instances in which we use jQuery are not grounded in a lack of functionality
in Comb; Our goal has never been to replace jQuery.
In \ref{sec:simplifying} we did however mention the desire to fit Comb into an
ecosystem of existing software. Being able to wrap properties of our dataset
with a jQuery object to empower it with additional features is a step in exactly
that direction.

However, Comb does in no way require the developer to use jQuery for such tasks.
Tools such as prototype.js\footnote{\url{http://prototypejs.org/}} and
zepto.js\footnote{\url{http://zeptojs.com/}} may of course be used instead.
By allowing diversity like this we follow another rule set out in
``The Art of Unix Programming'':
\begin{citequote}{\cite[Chapter 1]{UXART}}
Rule of Diversity: Distrust all claims for ``one true way''.
\end{citequote}

The need for us to implement an \inline{editable()} function highlights the
price we pay for this diversity. We can currently not bind properties of
backbone models to values in the dataset as easily as we intended to do at the
conclusion of chapter \ref{chap:requirements}. In chapter \ref{sec:revised-goal}
we revised that goal for a reason we can state more clearly now:
\begin{quote}
Imbuing Comb with the ability to detect changes in the DOM via jQuery and
interfacing directly with the backbone API forces us to forego integration with
any other type of framework or tool.
\end{quote}
A more sensible approach to such an integration would be a separate layer which
handles communication between Comb and whichever other library we wish to
integrate with. Such a layer has the possibility to choose libraries that
enhance Combs features in a way specific to its purpose.

The developer from the scenario in chapter \ref{sec:scenario} for example has
no need for an entire client-side Model-View-Controller framework.
For him a simple form layer that retrieves values and submits them as JSON to a
URL would suffice.


\subsection{Template Editor}
\todo{Write about how Comb can bring something to the mix even without a server
Hook onto the previous chapter and introduce stuff by refering to parentNode,
which is the main reason for using Comb.}


\section{Reviewing our goal}
In the requirements for the prototype (chapter \ref{chap:requirements})
we set out to couple the dataset fed into the server-side template engine with
the models on the client-side. In \ref{sec:revised-goal} we revised this
requirement by splitting the task into two distinct actions and changed the goal
to implementing the first part only: Extracting the dataset from a rendered
template.

The architecture we set to achieve this goal aimed at a two parted process where
we retrieve information from templates and later use that information to parse
rendered templates. The demo applications from chapter \ref{chap:demo}
integrate Comb to do exactly that. They render templates with given a dataset,
extract the data from the rendered templates and employ that data in the rest of
the application.

By keeping the footprint of our application low we have also increased the
maintainability of our tool. As stated in chapter \ref{chap:revised} we managed
to create a ``nucleus of code, which only incorporates the parts that are
necessary and unique to our solution''. Through increased maintainability we
follow another rule set out by Eric Raymond:
\begin{citequote}{\cite[Chapter 1]{UXART}}
Rule of Extensibility: Design for the future, because it will be here sooner than you think.
\end{citequote}
To extract data, the Comb client does not depend on any client libraries, which
may need upgrades or become superseded by better projects.
We achieved this not by incorporating functionality from other libraries
directly into the code of the client, but by restricting the feature-set of Comb
to the core functionality a ``nucleus of code''.




\section{Pre-Parser tool improvements}
As with any piece of software, there is always room for improvement.
Comb is no different. In this section we will look at which parts of the
pre-parser can be improved.

\subsection{Parser}

\subsubsection{Extending the template grammar}
Looking at the Mustache-XML EBNF\footnote{figure \ref{fig:mustache-xml.ebnf}} we
can see how the grammar is simplistic, when compared with the
full XML \cite[section 2/\#sec-documents]{XMLSPEC} EBNF or to
HTML \cite{HTMLSPEC}\footnote{
	HTML 5 has no EBNF, the reason for that is detailed at
	\url{http://lists.w3.org/Archives/Public/www-tag/2009Sep/0013.html}
}. The intention is not to build a fully capable HTML parser, but the question
remains whether our grammar matches a superset which is generic enough to allow
for all templates that can be converted into a DOM. Among the possible problems
that may be encountered is namespacing of tag names. The \inline{identLetter}
\cite{PARSECDOC}\footnote{
	Go to: /doc/html/Text-Parsec-Token.html\#v:identLetter
} property of our XML token parser in our
pre-parser tool lists the allowed letters in an identifier,
which in our case are xml tag names and attribute names. Among those letters is
the colon. Comb should therefore be able to recognize namespaced tag names
since the DOM API returns their name with the namespace included.

\subsubsection{Standalone mustache tag lines}
As we describe in \ref{sec:standalone-lines} mustache removes
lines containing only white space and a mustache section tag.
The demo application runs a modified version of mustache.js where this feature
is removed\footnote{
	Specifically we removed the highlighted line seen at
	\url{https://github.com/janl/mustache.js/blob/master/mustache.js\#L512}
}. Our parser does currently not account for this detail. This causes
our client library to expect a whitespace line where in fact there is none.

\subsubsection{Mustache comments}
\label{sec:comment-selectors}
In \ref{sec:mustache-xml-ebnf} we deemed a mustache comment not relevant
``since it does not output any content our client library can retrieve''.
In \ref{sec:parent-nodes} we introduced parent nodes into the dataset we return
for variables.
This addition makes comments useful even though it does not output any data.
By placing comments in key locations in a template, the developer can access
nodes in a rendered template by referencing those comments and accessing their
parent node. Because comments do not affect the rendered template we can even
adjust our EBNF to accommodate comments in all parts of a template.
This includes the space between XML attributes.

Comments do however not contain identifiers but free form strings.
To simplify referencing we could consider anything up to a set of
separators as the identifier. The alternative would be to simply require the
developer to specify the entire string.

Although this feature addition can be considered a significant improvement to
the possibilities of our tool, it can not be deemed a shortcoming.
Our goal was to retrieve the original dataset from a rendered template.
Comments do not receive any dataset value and do not output any value, as such
they were out of scope.

\subsection{Resolver}

\subsubsection{Identifier origins}
In section \ref{sec:arch-comp-tpl} we determined that we would not be able to
determine the origin of a key in the dataset precisely, when the context
stack is greater than one. This holds true for all cases except the inverted
section. The inverted section does not push a new context on top of our existing
context stack because it is only rendered when its identifier points at an empty
list\footnote{
	Or the data it points at is coerced into an empty list
} in the dataset. This is guaranteed by the specification\footnote{
	Read: ``This section MUST be rendered if and only if the data list is empty.''
}:
\begin{citequote}{\cite[inverted.yml]{MSTSPEC}}
This section MUST NOT be rendered unless the data list is empty.
\end{citequote}
We can be certain that all identifiers referenced in an inverted section do not
belong to the identifier of the section. This fact allows us to lift the parsed
values into the parent context.

\subsubsection{Lookahead}
\label{sec:lookahead}
The filter in our pre-parse tool currently requires mustache tags to be
separated by strings or XML tags. This restriction is necessary to ensure that
our client library can recognize the end of variables as well as the beginning
and end of section iterations. The parser employs a limited form of lookahead
when it checks whether an iteration is followed by another iteration. Variables
consider stop parsing as soon as they encounter the first occurrence of their
next sibling. This is a rather limiting restriction which can hinder developers
in writing templates.

We can overcome this restriction by enhancing the parsing capabilities of our
client library. From the pre-parser tool we only need to remove the
\emph{no\_lookahead} function located in the list of filters.

To create a simplistic form of lookahead we can catch errors thrown by
mustache tag objects when they are unable to verify their siblings and try
alternative possibilities until we succeed. This is however not only resource
intensive but may result in a faulty dataset extraction, because more than one
combination of possibilities can be applied to a DOM tree.

A more sound approach requires extending our pre-parser tool as well. It can
combine the same aforementioned possibilities of parsing a template and replace
any unknowns (i.e. variables) with wildcard characters our client library can
recognize. Whether a section is entered is also not knowable when
analyzing a template, but we can create a binary tree of possibilities,
where each node represents a list of XML tags. Upon encountering a section we
branch and continue the tree generation. This tree generation can continue until
we encounter the end of a DOM tree. We can also set a limit on the depth of the
tree, thereby limiting the amount of possibilities our client library has to
try. Once the tree is generated the filter may analyze it and print errors if
two paths are indistinguishable. The tree can then be reduced in size without
reaching ambiguous paths.

The client library can for example use this tree to determine whether a
section containing a variable as its first child and a variable as its
next sibling has another iteration by analyzing the content that follows this
ambiguity.

Such an improvement would also obsolete our filter that checks whether the
first child and next node of a section can be confused
(\emph{ambiguous\_boundaries}, section \ref{sec:filter}).

\subsubsection{Variable boundary ambiguity}
\label{sec:var-boundary-ambiguity}
Our lookahead improvement can not help alleviate a similar problems a developer
may encounter with two or more variables in one text node as shown in figure 
\ref{fig:twovars-no-sep}.
\begin{figure}
	\centering
	\begin{subfigure}{\textwidth}
		\caption{Template}
		\label{fig:twovars-no-sep.mustache}
		\begin{lstlisting}[language=mustache]
<span>{{var_one}}{{var_two}}</span>
		\end{lstlisting}
	\end{subfigure}
	
	\begin{subfigure}{\textwidth}
		\caption{Result}
		\label{fig:twovars-no-sep.html}
		\begin{lstlisting}[language=HTML]
<span>This text can be split between the two variables or belong to only one.</span>
		\end{lstlisting}
	\end{subfigure}
	\caption{Two variables without a separator between them}
	\label{fig:twovars-no-sep}
\end{figure}
The parser cannot split a string originating from two variables that are not
separated by text in any meaningful way. It is impossible to determine how much
text belongs to the first variable and how much text belongs to the second
variable\footnote{This also applies to more than two variables of course.}.

Separating variables with text can alleviate this problem only if the variable
values do not contain the separator itself. Consider the example in figure
\ref{fig:twovars-sep}, here the library cannot split the string in any
meaningful way either. We know that \inline{var\_two} at least contains
``slashes'', because there is no slash after the second variable in the
template. We can also determine that \inline{var\_one} at least contains
``path'', because we would see two subsequent slashes if \inline{var\_one} was
empty. Of the remaining slashes in the ``/with/multiple/'' string the
original separator can be any one of them.
\begin{figure}
	\centering
	\begin{subfigure}{\textwidth}
		\caption{Template}
		\label{fig:twovars-sep.mustache}
		\begin{lstlisting}[language=mustache]
<a href="http://www.example.com/{{var_one}}/{{var_two}</a>
		\end{lstlisting}
	\end{subfigure}
	
	\begin{subfigure}{\textwidth}
		\caption{Result}
		\label{fig:twovars-sep.html}
		\begin{lstlisting}[language=HTML]
<a href="http://www.example.com/path/with/multiple/slashes</a>
		\end{lstlisting}
	\end{subfigure}
	\caption{Two variables separated with a `/'}
	\label{fig:twovars-sep}
\end{figure}

\subsection{Filter}

\subsubsection{Suggestions}
Thanks to the Parsec library, the developer receives helpful messages when a
template cannot be parsed. Our filter component also outputs detailed error
messages. The errors can be improved by accompanying them with suggestions for
how to resolve them. Filters have access to enough information to calculate how
an error may be avoided.

\subsection{Generator}

\subsubsection{Notification of changes}
In \ref{sec:assessment} we highlighted how the Movie Database \#2 Comb version
needed no modifications to the identifiers used to access values in the dataset.
In some cases this may however be necessary, since a change may remove
a variable or section entirely from the template. Identifiers may be renamed or
part of the template is factored out into a partial.

We can warn the developer of such changes if we are asked to overwrite an
existing comb file. Assuming the file we are about to overwrite originates from
the old template, we can parse its content and compare it with the newly
generated content. Armed with this information, the pre-parser tool can alert
the developer to any changes between two versions of a template that require the
code accessing the extracted dataset to be changed.

More importantly, the developer will also be aware when this is \emph{not}
necessary. If a regenerating a comb file produces no alerts the developer can
be \emph{certain} that no client-side code needs adjustment.




\section{Client library improvements}

\subsection{Mustache tags as tag and attribute names}
In \ref{sec:template-constraints} listed the inability to specify mustache
tags in XML tag names and attribute names as a restriction on the templates our
tool can parse. We may however be able to allow this by changing our grammar and
modify our resolution phase to link mustache tags in a more generic way.
Our client library addresses DOM elements by child node indexes instead of tag
names and can therefore find these tags with no modifications. Tag names are
however used to determine various aspects of mustache tag boundaries. These
checks will have to be rewritten to allow for variable tag names.

\subsection{Fall back to lambda sections}
Lambda sections may currently cause unexpected errors in our client library.
It assumes every section is a normal section and throws errors if it does not
find what is expected to be found. Instead of stopping the parsing process
the client library should fall back to the assumption that the current section
is a lambda section and collect its contents. Similar to ambiguous variable
boundaries this strategy may be problematic as well. The parser can in many
cases not be sure when a lambda section is actually finished because its content
can vary wildly.

\subsection{Partials must be only children}
\label{sec:partial-only-child}
The filter of our pre-parser tool requires partials to be contained within an
XML tag as its only child. This is done to simplify the parsing process, which
would otherwise be complicated by accounting for siblings of root nodes in
a template.

\subsection{Unescaped variables as last children}
\label{sec:unescaped-variable-filter}
Much like lambda section the content of unescaped variables is hard to predict
and can be confused with the next sibling of the variable. We added the filter
\emph{unescaped\_pos} in the filter component of our pre-parser tool for that
reason. With the guarantee that an unescaped variable is always the last child
of an XML tag, our client library can add all nodes it encounters to the value
list of an unescaped variable beginnning at the location of that variable until
it finds no more nodes.

\subsection{Properties of values}
\label{sec:flatten-section}
When a property of an identifier is accessed, we know this identifier points at
an object rather than a list. If we in the same template encounter a section
using the same identifier, we can recreate the original dataset more precisely
by having the section return an object instead of a list with a single item
containing that object.

\subsection{Improving the retrieved dataset}
\todo{This is ingenious and fucking important, read and correct multiple times}
A more general approach to improving the retrieved dataset is to introduce
ambiguity. The developer may be more familiar with the structure of the dataset
passed to the template engine than the structure of the template, which dictates
the structure of the dataset retrieved from the rendered template.

When a section intended as an if block is retrieved from the rendered template
the client library returns an array regardless of the amount of iterations the
section has performed\footnote{
	not accounting for effects of the modification proposed in \ref{sec:flatten-section}
} (which in the case of if blocks should be zero or one).
Since the original dataset held a single object at that position,
the developer will not expect to find an array.

Our tool has no way of determining the semantic meaning of a section in a
template.
Fortunately we do not need to be certain what the intended usage of the section
is to rebuild the original dataset structure.
If a section iterated only once we can decorate the resulting array with every
property of the first entry in the array. Whichever version the developer
expected, he will now be able to access it that way. This approach will however
not work when there are no iterations since we cannot let a key in our dataset
be both an empty list and the value false\footnote{
	Double negating an array in JavaScript (\inline{\!\![]}) returns true instead
	of false unfortunately (for our use-case)
}. This means the user will not be able to check for the existence of an object.




\section{Future Work}

\subsection{Modifying rendered templates}
\label{sec:update-dom}
The \inline{update()} function\footnote{section \ref{sec:update}} returned by
variables is helpful when the developer intends to update strings in the
template. In \ref{sec:loopbacking} we demonstrated how our demo application
could ``push'' and ``pop'' iterations onto and from a dataset and then re-render
the template with the new values to display the changes.
This re-rendering performs a lot of unnecessary work considering it is only
the rendered template content of one section that needs to be appended or
removed.

If our pre-parser tool were to extract the template content of a section, we
could append an additional iteration by only rendering this content with the
newly pushed item using \emph{mustache.js}.
If we are to remove an iteration we simply remove the nodes that were recognized
as belonging to that iteration when the rendered template was parsed.

\subsection{Two-way binding of models}
\label{sec:two-way-binding}
After implementing \ref{sec:update-dom} layers can be constructed as
mediators between Comb and client model libraries. As an example of such a
library we will choose backbone. The backbone library features an event
architecture which allows the developer to be notified when properties of
objects change. Using this architecture we can create a layer that allows the
developer to specify which identifier in the dataset returned by Comb a
backbone model property corresponds to. Lists in the dataset may be mapped to
collections.

Once such a mapping is established, our layer can propagate changes in the
models to the DOM. Reversing this effect is also possible by listening for
changes in the DOM (e.g. a change of an input field value) and updating the
mapped model instead.

Such a layer implements the feature we originally set out with implementing as
well, but chose to forgo when we in chapter \ref{chap:revised} revised the
requirements of our tool.

A feature like this will surely also encourage the use of Comb even if the
template was rendered on the client, making the retrieval of values from the DOM
a secondary objective.



\section{Emergent Properties}
\label{sec:emergent}
In \ref{sec:revised-goal} we outlined the rules from Eric Steven Raymonds book
``The Art of Unix Programming''. Although, the rules are meant for developing
Unix programs, we chose to apply them to the development of Comb.
Comb can not be regarded as a framework but a tool. It does not dictate what
software the developer should use (beyond mustache)\footnote{
	In fact, Comb is even agnostic about HTTP. How the comb file is transfered
	is up to the developer.
} and it does not expect a specific structural layout of a web application.
This fact allows developers to integrate Comb in a larger set of software
combinations, extending its applicability.
In fact, Comb may be used in ways we may not have outlined in this evaluation:
Node.js\footnote{See \ref{sec:nodejs}} allows execution of JavaScript on the
server side. With node.js Comb could be used in the parsing of rendered
templates without involving any browser\footnote{
	This could also be a great way to write unit tests for Comb.
}.

By choosing to develop Comb as a tool we can also observe a set of emergent
properties that have become apparent during the implementation, usage and
evaluation of it.

\begin{itemize}
\item Templates a verified for proper HTML syntax through the pre-parser tool.
\item Data extraction implies a one-way communication, but using
      \inline{parentNode} allows Comb to supply the developer with additional
      information, that can be used for various kinds of extensions.
\item The \inline{update()} function allows for some limited DOM manipulation.
      In \ref{sec:update-dom} we describe how this can be extended sections.
\end{itemize}

\end{document}
