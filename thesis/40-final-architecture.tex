\chapter{Final Architecture}
\section{Mustache}

\section{Overview}
\begin{itemize}
\item Use haskell with parsec
\item Generate information for the client
\item Architecture for retrieval of that information
\item Client library to put that information to use and retrieve values
\end{itemize}

\section{Strategy for retrieving values}
\begin{itemize}
\item Minimize scope (only loops and variables, no partials)
\item Always treat sections as loops. Even if they are if blocks
\item Recognize sections and variables by boundaries (inner & outer)
\item Create JSON objects containing enough information for the client library to retrieve values
\item Don't bother with scoping, let the user decide when a variable is actually scoped
\end{itemize}

\section{Challenges}
\begin{itemize}
\item Unescaped variables can wreak havoc
\item Ambiguities can occur between two boundaries
\item TextNodes that are separate in the parser are merged on the client
\end{itemize}
