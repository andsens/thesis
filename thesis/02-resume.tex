% !TEX root = thesis.tex
\documentclass[thesis.tex]{subfiles} 
\begin{document}

\chapter*{Resum\'e}
\label{chap:resume}
\addcontentsline{toc}{chapter}{Resum\'e}

Det er et langsommeligt arbejde at indhente informationer fra HTML og
knytte \emph{event handlers} til \emph{Document Object Model} (DOM) i
web applikationer via \emph{CSS selectors}.
Der har været mange forsøg på at rette op på den grundlæggende årsag til dette,
men også nogle forsøg der afhjælper symptomerne i stedet.
jQuery er det mest velkendte blandt disse forsøg.
Vi præsenterer et værktøj der løser netop dette problem.
Det udnytter web applikationers brug af HTML skabeloner (\emph{templates}),
som kan analyseres for at automatisere navigationen gennem DOM'en
--- en førhen manuel opgave.
Desuden viser vi hvordan en klar afgrænset, minimal omfang af værktøjer
kan øge deres anvendelighed og hjælpe dem med at trives i et økosystem af
anden software.

Vores værktøj er baseret på \emph{mustache template engine}; et template sprog,
der er implementeret i mange forskellige programmeringssprog.
Værktøjet genskaber på klienten det oprindelige datasæt der blev brugt til at
rendere et template.
Dette rekonstruerede datasæt gør det muligt for en udvikler at læse værdier
fra --- og interagere med --- DOM'en ved at bruge de i forvejen kendte nøgler
der blev brugt som variable i skabelonerne.
Denne metode er også mere robust end at bruge CSS selectors,
fordi navnene på nøglerne ikke ændrer sig når templates omstruktureres.
Vi demonstrerer dette med to web applikationer.

\end{document}
