% !TEX root = thesis.tex
\documentclass[thesis.tex]{subfiles} 
\begin{document}

\chapter{Architecture}
\label{chap:arch}

Our goal is to create a tool that allows the user to pass a rendered template
(DOM) to it and receive a data structure that equates
the original dataset passed to the template engine.
Parsing can be a process that requires a lot of processing power. We do not want
our tool to slow down the web application every time a new template is rendered
and the values are retrieved on the client. To minimize the effort required to
parse a rendered template we try to compile as much information about a template
as possible before it is rendered. Using this information we should be able
to parse a rendered template more quickly.
This strategy implies a pre-parsing step that outputs data which aids
the client library in the parsing process. Since this is an operation
that only needs to be run once before a web application is deployed, we are not
constrained by the environment the actual web server runs in.

\section{Compiling template information}
We need to be able to parse a mustache template and acquire the necessary
information to enable the client library to parse a rendered template.
We analyze the capabilities of the mustache template language and arrive
at the following pieces of information, which we can deduce before
the template is rendered.

\begin{itemize}
\item The location of variables in the document
\item Whether a variable is escaped or unescaped
      (\inline{\{\{identifier\}\}} vs. \inline{\{\{\{identifier\}\}\}})
\item The location of sections in the document
\item Whether a section is inverted
      (\inline{\{\{\#identifier\}\}} vs. \inline{\{\{^identifier\}\}})
\item The contents of a section
\item The location of partials in the document
\item The location of comments in the document
\end{itemize}

Because of the nature of mustache templates we can however not retrieve the
following data:

\begin{itemize}
\item The contents of variables (including their type, e.g. integer, string)
\item The number of iterations a section will run
\item Whether a section is a loop or an if block
      (except in the case of an inverted section)
\item Whether a section is a lambda section or an actual section.
\item The behavior of a lambda section
\item The template a partial points at
\item Whether an identifier refers to a key in the current stack level or
      to a key in one of the lower stack levels
\end{itemize}

The rendered template is in the Document Object Model format, once it has
been rendered by the client. We can still access the rendered template as a
string after it has been inserted by accessing the \inline{innerHTML} property
on the element node it was inserted into. This has one major drawback.
The browser does not return the actual string that was inserted but rather a
serialized version of the DOM. This is demonstrated quite easily by executing
the following lines in the Google Chrome Developer Console:
\begin{lstlisting}[language=HTML]
var div = document.createElement("div")
div.innerHTML = "<img/>"
div.innerHTML
\end{lstlisting}

The last line does not return the string "\ <img/\ >" but "\ <img\ >".
In Firefox the last line returns "\ <img xmlns="http://www.w3.org/1999/xhtml" /\ >".
This means that the rendered templates cannot be parsed reliably by using the
\inline{innerHTML} property.

However, no matter which way the browser decides to represent an HTML tag,
we can still rely on the ordering of tags and on the names of tags to be the same
(case sensitivity can be avoided with a simple
\inline{element.tagName.toLowerCase()} when comparing tag names client side).
We therefore opt to relate the information about mustache tags to the DOM instead.

This choice requires our pre-parsing tool to be able to understand
HTML documents and construct a data structure resembling the DOM in which the
mustache tags can be located and their location converted into a DOM path.

\section{Communicating with the client}

The information gathered by the pre-parser is saved in files next to the
original templates. The user may choose whatever technology fits best to
transfer this information to the client library. All the library should expect
is a DOM node with the rendered template as its children and the information
created by our pre-parser.

By choosing this approach we leave the user with many optimization
possibilities.
\begin{itemize}
\item Gzip templates and template information statically to optimize
      server performance.
\item Prepend template information to the template and subsequently
      cut it out before passing the template to the template engine.
      This way only one file needs to be handled and transferred.
\item Use require-js to load templates and their information in parallel
      while developing and subsequently inline both template and
      template information into one big JavaScript file when deploying.
\end{itemize}

Figure \ref{fig:architecture} illustrates our resulting architecture.
\todo{Explain the image}

\begin{figure}
	\centering
	% !TEX root = ../thesis.tex
\begin{tikzpicture}[every edge/.style={link}]
	\node[entity] (cdata) at (9,0) {Data};
  \node[relationship] (parser) at (6,0) {Parser} edge [->] (cdata);
	\node[entity] (tplinfo) at (0,0) {Tpl. info} edge [->] (parser);
	
	\node[entity] (dom) at (6,6) {DOM} edge [->] (parser);
  \node[relationship] (mustache) at (3,6) {mustache} edge [->] (dom);
	\node[entity] (tpl) at (0,6) {Templates} edge [->] (mustache);
	
  \node[relationship] (pre) at (0,3) {Pre-Parser} edge [->] (tplinfo);
	\node[entity] (sdata) at (3,3) {Data} edge [->] (mustache);
	
	\draw (tpl) -- (pre);
	
	\node[draw=black,thick,inner sep=2mm,rectangle,fit=(tplinfo) (tpl)] (static) {};
	\node[anchor=north,inner sep=1mm] at (static.south) {Static content};
	
	\node[draw=black,thick,inner sep=2mm,rectangle,fit=(mustache) (parser) (cdata)] (dynamic) {};
	\node[anchor=north,inner sep=1mm] at (dynamic.south) {Dynamic content};
	
	\begin{scope}[on background layer]
		\node[fill=gray!20,inner sep=2mm,rectangle,fit=(mustache) (tpl) (tplinfo)] (server) {};
		\node[anchor=south,inner sep=1mm] at (server.north) {Server};
		
		\node[fill=gray!20,inner sep=2mm,rectangle,fit=(dom) (parser) (cdata)] (client) {};
		\node[inner sep=1mm] at (client) {Client};
	\end{scope}
\end{tikzpicture}

	\caption{Architectural diagram of our tool}
	\label{fig:architecture}
\end{figure}

\section{Alternatives}

Instead of using a pre-parser, we could also take other paths to help
the client library in parsing rendered templates.

\subsection{Integrating with the template engine}

In chapter~\ref{chap:requirements} we touched upon the fact that it was really
the template engine that discarded the information and only outputted the
rendered template. We could choose to modify the template engine to output that
exact information. This poses a rather big challenge:
For mustache templates there exists no such thing as \emph{the} template engine.
Currently the mustache website lists mustache engines in 29 different
programming languages. (https://github.com/defunkt/mustache/wiki/Other-Mustache-implementations \todo{ref})
This fact makes the goal of such an approach very hard to achieve.

\subsection{Decorating templates}

The user could decorate mustache template tags with specific HTML tags that
have no effect on the visual layout but can be retrieved by the library.
This would make the client side code a very lightweight value retriever
thereby obsoleting the pre-parser.

Apart from simply shifting the workload of locating template variables
from the viewmodel maintainer to the template maintainer\footnote{The scenario
in chapter \ref{chap:requirements} illustrates this point},
the task of retrofitting existing web applications becomes much greater.

\end{document}
