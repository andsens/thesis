% !TEX root = thesis.tex
\documentclass[thesis.tex]{subfiles} 
\begin{document}

\chapter*{Abstract}
\label{chap:abstract}
\addcontentsline{toc}{chapter}{Abstract}

Extracting information from HTML and binding event handlers to the
Document Object Model in web applications using CSS selectors is a tedious
exercise. There have been many attempts to both rectify the root cause and
alleviate the symptoms with jQuery being the most prominent among them.
We present a solution which targets this precise problem and leverages the fact
that web applications use HTML templates, which can be analyzed to automate
the otherwise manual navigation through the DOM.

We also show how a clear-cut, minimal scope can increase the applicability of
a tool and allow it to thrive in an ecosystem of other software.

\todo{Jan: jQuery er et eksempel for symptom alleviation, ved ikke om der er behov for et mht. root cause}

\end{document}
