% !TEX root = thesis.tex
\documentclass[thesis.tex]{subfiles} 
\begin{document}

\chapter{Evaluation}
\label{chap:eval}
\section{Limitations}
\begin{itemize}
\item comment nodes useful, should not have been discarded
\item We don't get the entire section structure of a template unless all
      sections have at least one iteration
\item Cannot push or pop iterations without rerendering
\item Lambdas
\item Choice of parsing method
\item Client
\item Parsing problems on the client
\item Filter on compiler
\item Bug in mustache.js (not a bug, it's the spec)
      https://github.com/mustache/spec/blob/master/specs/sections.yml\#L192
      https://github.com/janl/mustache.js/blob/master/mustache.js\#L512
\item Removal of whitespace lines
\item Full XML EBNF a bit out of scope: http://www.jelks.nu/XML/xmlebnf.html
\item \& Does not throw proper error when not closed
\item Lexeme token parser versus normal token parser
\item Comment not working
\item variable text boundary recognition not perfect
\item inverted sections don't iterate, no need to filter
\item and no need to return array
\item If-else section return lists with single entry
\item Mustache identifiers can access more than the keys in the current context,
      they may also drill down and access properties of objects in the current
      context.
\end{itemize}

\section{Challenges}
\begin{itemize}
\item Unescaped variables can wreak havoc
\item Ambiguities can occur between two boundaries
\item TextNodes that are separate in the parser are merged on the client
\end{itemize}
\section{Advantages}
\subsection{Comparison}

\end{document}
