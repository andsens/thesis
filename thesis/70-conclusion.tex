% !TEX root = thesis.tex
\documentclass[thesis.tex]{subfiles} 
\begin{document}

\chapter{Conclusion}
\label{chap:conclusion}
Comb is a tool to extract the original dataset fed into the Mustache template
engine from the rendered template. We have over the course of this thesis made a
case for why we expect Comb to fit into the existing ecosystem of tools for web
application development. We began with a prototype that incorporated not only
the dataset extraction but also the coupling of said dataset to a client-side
Model.
This coupling led to the realization that the applicability of Comb can be
greatly increased by focusing on the data extraction and leaving the utilization
of that data up to the developer, who uses our tool.

Comb was split into two parts: the pre-parser tool and the parser running on
the client-side. This split freed us to implement the pre-parser in Haskell
and keep the client library dependency free.

We contrasted Comb with existing solutions and found that most of these require
the developer to switch to new templating languages explicitly made for the
problem they are trying to solve. We identified this strategy as a framework
approach, which is a sound approach to the problem, but brings with it a reduced
applicability. The framework approach is a general tendency we have discovered
in academic papers regarding web applications and templating languages. With
Comb we have shown an alternate path to a solution, which can be of use in
future works regarding this field.

Our contribution, Comb, is a \emph{tool} that enables developers to
build web applications with a more robust access to data embedded in rendered
templates. Comb does not restrict the developer to any kind of architectural
pattern and is built to be extended by other libraries and tools.

\todo{write}
\todo{Write that this approach may in fact be new}
\end{document}
